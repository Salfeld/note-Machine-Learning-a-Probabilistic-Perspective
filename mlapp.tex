
% \documentclass[11pt,singlecolumn,letterpaper,UTF8]{article}
\documentclass[11pt,a4paper]{article}
% \usepackage{naaclhlt2013}


%%%%%%% For Chinese Char %%%%%%%%%%%%%%%%%%%%%%
% \usepackage{xltxtra} % can use only once

% use for various fonts
\usepackage{xeCJK}
\setCJKmainfont[BoldFont={SimHei}]{SimSun}
\setCJKsansfont{SimSun} % Sorry I don't have enough fonts.
\setCJKmonofont{SimHei}

\punctstyle{quanjiao}

%%%%%%%%%%%%%%%%%%%%%%%%%%%%%%%%%%%%%%%%%%%%%%%

\usepackage{tabularx}
\usepackage{longtable}
\usepackage{color}
\usepackage{colortbl}
\usepackage{graphicx}
\usepackage{latexsym}
\usepackage{makeidx}

% for algorithm
\usepackage{algorithm}
% \usepackage{algorithmic}

\usepackage{graphicx}
\usepackage{bm}
\usepackage{xcolor}
\usepackage{pstricks}
\usepackage[noend]{algpseudocode}
\usepackage{cancel} % zdq@2012.05.30 for cancel in math formulas


% WARNING: hyperef can be used only once!!!
% \usepackage[bookmarks]{hyperref} % for bookmarks/index/idx in pdf 
% \usepackage[pagebackref=true,breaklinks=true,letterpaper=true,colorlinks,bookmarks=false]{hyperref}
\usepackage[bookmarks,pagebackref,breaklinks,colorlinks,urlcolor=blue]{hyperref} % zdq@2012.06.05 for backref


% \usepackage[margin= 4cm]{geometry}

% \usepackage{graphicx}
% \usepackage{epsfig}
% \usepackage{cancel}

% \usepackage[utf8x]{inputenc}


% \DeclareMathOperator*{\argmax}{arg\,max}
% \DeclareMathOperator*{\argmin}{arg\,min}

% \newcommand{\E}{{\mathbb{E}}} % zdq@2012.05.29 for Expectation's E in Math formula

% Setup TikZ
% \usepackage{tikz} % zdq@2011.09.18 for draw
% \usetikzlibrary{arrows}
% \tikzstyle{block}=[draw opacity=0.7,line width=1.4cm]


% \def\ie\emph{i.e.}

% \linespread{1.5}


% for math and formula
\usepackage{amsfonts}
% \usepackage{amssymb}
\usepackage{amscd}
\usepackage{amsmath}
\usepackage{amsthm} % for new theorem 
\newtheorem{Objective}{Objective}
\newtheorem{Assumption}{Assumption}
\newtheorem{Estimator}{Estimator}

\newcommand{\iref}[1]{~{\ref{#1}}~} % zdq@2012.05.29 for ~\ref{}~
\newcommand{\icite}[1]{~{\cite{#1}}~} % zdq@2012.05.29 for ~\cite{}~
\newcommand{\engtxt}[1]{{\texttt{#1}}} % zdq@2012.05.29 for English text in Chinese env
\newcommand{\E}{{\mathbb{E}}} % zdq@2012.05.29 for Expectation's E in Math formula
\newcommand{\R}{{\mathbb{R}}} % zdq@2012.05.29 for Real Field's R in Math formula

\DeclareMathOperator*{\argmax}{arg\,max}
\DeclareMathOperator*{\argmin}{arg\,min}
% \usepackage[normalem]{ulem} % zdq@2011.12.01 for underline; fail to compile under xelatex

%% Setup TikZ
\usepackage{tikz} % zdq@2011.09.18 for draw
% \usetikzlibrary{snakes}
\usetikzlibrary{arrows}


% for Chinese
\usepackage{xltxtra}
% \setmainfont[Mapping=tex-text]{WenQuanYi Micro Hei}

% \AtBeginDvi{\special{pdf:tounicode UTF8-UCS2}} % make side index not Garbled


\title{Note: Machine Learning a Probabilistic Perspective \cite{murphy:2012:Book:MLaPP} by Murphy }

\author{\textit{Zheng Daqi}}
\makeindex
\begin{document}
\maketitle

\abstract{This is a note taken when I read the book}\\

\section{Reading Plan and Progress}
\begin{itemize}
 \item Contents -- Done
 \item Chapter 1 -- today 3:00--5:00 pm
 \item Every Chapters' Summary -- today
\end{itemize}

\section{Notation Explanation}
\begin{itemize}
 \item Most of the contents are the direct quotation from the book, so without extra notation.
 \item Sections started with ``My'' are my notes.
 \item Items marks by ``TODO'' are my notes.
 \item Anything braced in square braces [my notes] are my notes.
\end{itemize}


\section{Concepts (Definition)}
\paragraph{Machine Learning}\label{Definition:Machine Learning}
we define machine learning as a set of methods that can
automatically detect patterns in data, and then use the uncovered patterns to predict future
data, or to perform other kinds of decision making under uncertainty (such as planning how to
collect more data!).

\subsection{Types of Machine Learning \ref{Definition:Machine Learning}}
\paragraph{Supervised Learning}\label{Supervised Learning}
(predictive)
\subparagraph{Goal}
the goal is to learn a mapping from inputs $\mathbf{x}$ to outputs $y$.
\newline
Our main goal is to make predictions on novel inputs, meaning ones that we have not seen before (this is called generalization), 
since predicting the response on the training set is easy (we can just look up the answer).
\subparagraph{Given}
given a labeled set of input-output pairs $\mathcal{D}=\{(\mathbf{x}_i,y_i)\}^N_{i=1}$

\paragraph{Unsupervised Learning}
(descriptive, knowledge discovery)
\subparagraph{Goal}
the goal is to find “interesting patterns” in the data.
\subparagraph{Given}
only given inputs $\mathcal{D}=\{\mathbf{x}_i\}^N_{i=1}$

\paragraph{Reinforcement Learning}
(not included in this book)
This is useful for learning how to act or behave when given occasional reward or punishment signals. 
(For example, consider how a baby learns to walk.)

\subsection{Tasks in Supervised Learning \ref{Supervised Learning}}

\paragraph{Classification}
Here the goal is to learn a mapping from inputs $\mathbf{x}$ to outputs $y$,where $y∈{1,...,C}$, with $C$ being the number of classes.
\begin{description}
 \item [binary classification] $C=2$
 \item [multiclass classification] $C>2$
 \item [multi-label classification] If the class labels are not mutually exclusive (e.g., somebody may be
classified as tall and strong), we call itmulti-label classification, but this is best viewed as
predicting multiple related binary class labels (a so-called multiple output model).
\end{description}

One way to formalize the problem is as function approximation.

\subsection{TODO}
\paragraph{Problem Formalization}
\paragraph{Exploratory Data Analysis}\label{Definition:Exploratory Data Analysis}

\section{Phenomenon}
\paragraph{Long Tail}\label{Phenomenon:Long Tail}
In fact, data across a variety of domains exhibits a property known as thelong tail, 
which means that a few things (e.g., words) are very common, 
but most things are quite rare (see Section 2.4.6 for details).

\section{Attitude}
\paragraph{Problem involving uncertainty} 
This books adopts the view that the best way to solve such problems is to use the tools
of probability theory. Probability theory can be applied to any problem involving uncertainty.

\paragraph{Generalizing from small data}
This \ref{Phenomenon:Long Tail} means that the core statistical issues that we discuss in this book, 
concerning generalizing from relatively small samples sizes, 
are still very relevant even in the big data era.

\paragraph{Feature Extraction} 
Such feature extraction is an important, but difficult, task. 
Most machine learning methods use features chosen by some human. 
Later we will discuss some methods that can learn good features from the data.

\section{Experience}
\paragraph{Exploratory Data Analysis}\ref{Definition:Exploratory Data Analysis}
It is always a good idea to perform exploratory data analysis, 
such as plotting the data, before applying a machine learning method.

\section{My Experience}

\subsection{What to Plot}
\paragraph{Scatter Plot}
Visualize the pair-wised variables relationship.

\section{My Idea TODO}
\paragraph{Re-Plot the 20 newsgroup classification in Figure 1.2 }
sort the words in different order.
\begin{itemize}
 \item total frequency
 \item other metric
\end{itemize}


\bibliographystyle{plain}

\bibliography{library}

\end{document}
